\section{Общие сведения о программе}

Для преобразования входного изображения в RGB-массив используется сторонняя библиотека stb\_image в виде двух заголовочных файлов. На вход программе подаются имена входного и выходного изображений, количество раз, которое надо наложить фильтр, размер \enquote{окна} и, опционально, количество потоков (по умолчанию 4). Программа делит изображение горизонтально на количество частей, равное количестве потоков, и обрабатывает эти части параллельно. На выходе получаем сглаженное изображение.

\pagebreak

\section{Общий метод и алгоритм решения}

Для реализации поставленной задачи необходимо:

\begin{enumerate}
    \item Изучить принципы работы pthread.
    \item Изучить работу с библиотекой stb\_image.
    \item Написать обработку аргументов запуска программы.
    \item Написать функцию разбиения изображения на части.
    \item Написать функцию запуска многопоточной обработки.
    \item Написать функцию медианного фильтра.
    \item Написать обработку ошибок
    \item Написать тесты
\end{enumerate}

\pagebreak

\section{Исходный код}

Код библиотеки stb\_image приводить не буду, так как это не относится к заданию лабораторной.

\textbf{lab3.c}

\begin{lstlisting}[language=C]

#include <pthread.h>
#include <stdio.h>
#include <stdlib.h>
#include <time.h>

#define STB_IMAGE_IMPLEMENTATION
#include "stb_image.h"
#define STB_IMAGE_WRITE_IMPLEMENTATION
#include "stb_image_write.h"

int min(int a, int b) {
  if (a < b) {
    return a;
  } else {
    return b;
  }
}

int max(int a, int b) {
  if (a > b) {
    return a;
  } else {
    return b;
  }
}

void swap(unsigned char **lhs, unsigned char **rhs) {
  unsigned char *tmp = *lhs;
  *lhs = *rhs;
  *rhs = tmp;
}

unsigned char *data = NULL;
int width = 0;
int height = 0;
int comp = 0;
int window_size = 0;

int thread_count = 4;

int filter_by_average(int left_border, int right_border, int lower_border, int upper_border) {
  int sum = 0;
  int count = 0;
  for (int i = lower_border; i <= upper_border; ++i) {
    for (int j = left_border; j <= right_border; j += comp) {
      sum += data[i * width * comp + j];
      ++count;
    }
  }
  return sum / count;
}

int compare(const void *x1, const void *x2) {
  return (*(int *)x1 - *(int *)x2);
}

int filter_by_median(int left_border, int right_border, int lower_border, int upper_border) {
  int size_gor = (right_border - left_border) / comp + 1;
  int size = (upper_border - lower_border + 1) * size_gor;
  int arr[size];
  int arr_idx = 0;
  for (int i = lower_border; i <= upper_border; ++i) {
    for (int j = left_border; j <= right_border; j += comp) {
      arr[arr_idx] = data[i * width * comp + j];
      ++arr_idx;
    }
  }
  qsort(arr, size, sizeof(int), compare);
  return arr[size / 2];
}

int filter(int idx) {
  int real_width = width * comp;
  int left_border = max((idx % real_width) % comp, idx % real_width - window_size * comp);
  int right_border = min(real_width - comp + (idx % real_width) % comp, idx % real_width + window_size * comp);
  int lower_border = max(0, idx / real_width - window_size);
  int upper_border = min(height - 1, idx / real_width + window_size);

  //return filter_by_average(left_border, right_border, lower_border, upper_border);
  return filter_by_median(left_border, right_border, lower_border, upper_border);
}

struct process_buf_args {
  unsigned char *buf;
  int from;
  int to;
};
typedef struct process_buf_args process_buf_args;

void *process_lines(void *args) {
  process_buf_args *ft = (process_buf_args *)args;
  for (int i = ft->from; i < ft->to; ++i) {
    for (int j = 0; j < width * comp; ++j) {
      int idx = i * width * comp + j;
      ft->buf[idx] = filter(idx);
    }
  }
  return NULL;
}

void process_image_to(unsigned char *result) {
  process_buf_args args[thread_count];
  int step = height / thread_count;
  if (step == 0) {
    fprintf(stderr, "The height of the image cannot be greater than the number of threads\n");
    return;
  }
  for (int i = 0; i < thread_count; ++i) {
    args[i].buf = result;
    args[i].from = step * i;
    args[i].to = step * (i + 1);
  }
  args[thread_count - 1].to = height;
  pthread_t ids[thread_count];
  for (int i = 0; i < thread_count; ++i) {
    pthread_create(&ids[i], NULL, process_lines, &args[i]);
  }
  for (int i = 0; i < thread_count; ++i) {
    pthread_join(ids[i], NULL);
  }
}

int main(int argc, char **argv) {
  if (argc < 5 || argc == 6 || argc > 7) {
    fprintf(stderr, "USAGE: %s <input_file> <output_file.png> <count> <window_size>\n", argv[0]);
    fprintf(stderr, "or\nUSAGE: %s <input_file> <output_file.png> <count> <window_size> -t <thread_count>\n", argv[0]);
    return 1;
  }
  if (argc == 7) {
    if (strcmp(argv[5], "-t") == 0 && atoi(argv[6]) > 0) {
      thread_count = atoi(argv[6]);
    } else {
      fprintf(stderr, "Bad thread arrgument\n");
      return 2;
    }
  }
  if (atoi(argv[3]) < 0 || atoi(argv[4]) < 0) {
    fprintf(stderr, "Bad argument 3 or 4\n");
    stbi_image_free(data);
    return 3;
  }

  data = stbi_load(argv[1], &width, &height, &comp, 0);
  if (!data) {
    fprintf(stderr, "Something was wrong during reading\n");
    stbi_image_free(data);
    return 4;
  }
  window_size = atoi(argv[4]);
  fprintf(stderr, "width:%d height:%d channels:%d threads:%d\n", width, height, comp, thread_count);

  time_t start = time(NULL);
  unsigned char *buf = (unsigned char *)malloc(width * height * comp * sizeof(unsigned char));
  for (int k = 0; k < atoi(argv[3]); ++k) {
    process_image_to(buf);
    swap(&data, &buf);
  }
  free(buf);
  time_t finish = time(NULL);
  fprintf(stderr, "time:%lds\n", finish - start);

  int res = stbi_write_png(argv[2], width, height, comp, data, width * comp);
  //int res = stbi_write_bmp(argv[2], width, height, comp, data);
  stbi_image_free(data);

  if (!res) {
    fprintf(stderr, "Something was wrong during writing\n");
    return 5;
  }
  return !res;
}

\end{lstlisting}

\pagebreak